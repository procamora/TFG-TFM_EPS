% !TeX document-id = {c6282027-a510-40f1-b05e-2f558a6cd3f1}
% !TeX program = xelatex
% !TeX TXS-program:compile = txs:///xelatex/[--shell-escape]
%%%%%%%%%%%%%%%%%%%%%%%%%%%%%%%%%%%%%%%%%%%%%%%%%%%%%%%%%%%%%%%%%%%%%%%%
% Plantilla TFG/TFM
% Universidad de Murcia. Facultad de Informática
% Realizado por: José Manuel Requena Plens
% Modificado: Pablo José Rocamora Zamora
% Contacto: pablojoserocamora@gmail.com
%%%%%%%%%%%%%%%%%%%%%%%%%%%%%%%%%%%%%%%%%%%%%%%%%%%%%%%%%%%%%%%%%%%%%%%%


% Archivo .TEX que incluye todas las configuraciones del documento y los paquetes. Añade todo aquello que necesites utilizar en el documento en este archivo.
% En él se encuentra la configuración de los márgenes, establecidos según las directrices de estilo de la EPS.
\input{include/configuracioninicial}

%%%%%%%%%%%%%%%%%%%%%%%%%%%%%%%%%%%%%%%%%%%%%%%%%%%%%%%%%%%%%%%%%%%%%%
% INFORMACIÓN DEL TFG
% Comentar lo que NO se desee añadir y sustituir con la información correcta.
%%%%%%%%%%%%%%%%%%%%%%%%%%%%%%%%%%%%%%%%%%%%%%%%%%%%%%%%%%%%%%%%%%%%%%
% Título y subtítulo
\newcommand{\titulo}{Título del Trabajo Fin de Grado/Máster}
\newcommand{\subtitulo}{Subtítulo del proyecto}
% Datos del autor
\newcommand{\miNombre}{Nombre Apellido1 Apellido2 (alumno)}
\newcommand{\miDNI}{00000000A}
\newcommand{\miEmail}{nombre@um.es}
% Datos del tutor/es
\newcommand{\miTutor}{Nombre Apellido1 Apellido2 (tutor1)}
\newcommand{\miTutorB}{Nombre Apellido1 Apellido2 (tutor2)}
\newcommand{\departamentoTutor}{Departamento del tutor}
\newcommand{\departamentoTutorB}{Departamento del cotutor}
% Datos de la facultada y universidad
\newcommand{\miFacultad}{Facultad de Informática}
\newcommand{\miFacultadCorto}{FIUM}
\newcommand{\miUniversidad}{\protect{Universidad de Murcia}}
\newcommand{\miUbicacion}{Murcia}
\newcommand{\firma}{include/firma}

% Configuración automática según el identificador elegido
% Grados
\definecolor{informatica}{RGB}{121,11,21}	% Informatica
% Colores generales
\definecolor{negro}{RGB}{0,0,0}
\definecolor{blanco}{RGB}{255,255,255}

% Logotipos comunes de todas las titulaciones
\newcommand{\logoFacultadPortada}{include/logos-titulaciones/Simbolo-negativo-sin-fondo}
\newcommand{\logoGradoPortada}{include/logos-titulaciones/Simbolo-negativo-sin-fondo} 
\newcommand{\logoUniversidadPortada}{include/logos-universidad/escudo_umu}
\newcommand{\logoFacultadPortadaBaja}{include/logos-titulaciones/FIUM-postivo-sin-fondo}
% Logos
\newcommand{\logoGrado}{include/logos-titulaciones/FI-positivo}
\newcommand{\logoDepartamento}{include/logos-titulaciones/Logotipo_DIIC}
\newcommand{\logoUniversidad}{include/logos-universidad/um}
% Texto
\newcommand{\miGrado}{Grado en Ingeniería Informática}
\newcommand{\tipotrabajo}{Trabajo Fin de Grado}
% Color
\newcommand{\colorgrado}{informatica}
\newcommand{\colortexto}{blanco}

% Información añadida a las propiedades del archivo PDF.
\hypersetup{
	pdfauthor = {\miNombre~(\miEmail)},
	pdftitle = {\titulo},
}

%%
% Archivo de acrónimos
%%
\makeglossaries % Genera la base de datos de acrónimos
\input{anexos/_acronimos.tex} % Archivo que contiene los acrónimos

%%%%%%%%%%%%%%%%%%%%%%%% 
% INICIO DEL DOCUMENTO
% A partir de aquí debes empezar a realizar tu TFG/TFM
%%%%%%%%%%%%%%%%%%%%%%%%
\begin{document}
	
	% Números romanos hasta el mainmatter.
	\frontmatter
	
	% PORTADA
	\input{include/portada/portada_color} % Portada Color
	\input{include/portada/portada_bn} % Portada B/N
	
	%%%%% PREAMBULO
	% Incluye el .tex que contiene el preámbulo, agradecimientos y dedicatorias.
	%\setstretch{1.5} % 1.5 línea de interlineado antes esta a 1 para la portada
	\input{preambulos/preliminaresconagradecimientos}
	\input{preambulos/1_declaracion_originalidad}
	%%%%%%%%%%%%%%%%%%%%%%%%%%%%%%%%%%%%%%%%%%%%%%%%%%%%%%%%%%%%%%%%%%%%%%%%
% Plantilla TFG/TFM
% Universidad de Murcia. Facultad de Informática
% Realizado por: José Manuel Requena Plens
% Modificado: Pablo José Rocamora Zamora
% Contacto: pablojoserocamora@gmail.com
%%%%%%%%%%%%%%%%%%%%%%%%%%%%%%%%%%%%%%%%%%%%%%%%%%%%%%%%%%%%%%%%%%%%%%%%


\chapter*{Resumen}

Aqui ira el resumen en del TFG

\todo[inline]{Falta por hacer}




Lorem ipsum dolor sit amet, consectetur adipiscing elit. Aliquam elementum, lorem at aliquet convallis, ipsum est elementum orci, eget venenatis quam nisi et mauris. Nullam dignissim lectus nec ante efficitur semper. Aliquam sed nisl lectus. Donec vel pellentesque orci. Ut et tempus neque, vel semper arcu. Morbi varius tortor vel posuere bibendum. Integer et pharetra eros. Ut efficitur malesuada rhoncus. Cras ac quam hendrerit, tempor eros vel, condimentum sapien. Integer in velit facilisis, rutrum enim vel, varius ante.

Donec nec nulla diam. Integer sed nulla at elit posuere suscipit non quis nulla. Pellentesque sit amet dictum eros. Vestibulum ante ipsum primis in faucibus orci luctus et ultrices posuere cubilia Curae; Sed dapibus pulvinar euismod. Donec ullamcorper cursus nunc quis ultrices. Cras ultrices dignissim justo at viverra. Suspendisse in felis enim. Mauris sodales vehicula laoreet. In suscipit dignissim augue eget laoreet. Sed sollicitudin fermentum turpis, ut malesuada nunc auctor nec. Nullam non justo cursus, dapibus nunc a, aliquet urna. Nulla semper tortor ut odio molestie tincidunt. Sed ut elit felis. Quisque turpis purus, suscipit vel gravida sit amet, laoreet sed massa.

In hac habitasse platea dictumst. Mauris sit amet pulvinar elit. Nam in neque gravida, imperdiet tortor sed, pellentesque diam. Quisque mattis orci ut felis aliquam malesuada. Ut a metus odio. Integer tincidunt, augue eu porttitor auctor, dolor tellus tincidunt dolor, cursus rhoncus ex leo in tellus. Donec est massa, consequat vitae nisi ut, posuere efficitur justo. Quisque non scelerisque purus. Nunc eu placerat magna. Sed facilisis metus vel pretium finibus. Sed eleifend bibendum dictum. Aenean odio odio, varius sit amet neque at, porttitor gravida mauris. In sit amet nisl magna.

Nulla metus nisi, faucibus volutpat vestibulum ut, efficitur a velit. Nam rhoncus dolor sit amet luctus pharetra. Nulla rutrum nibh at ullamcorper tincidunt. Cras et lectus ultrices, iaculis mi sit amet, lacinia nisi. Etiam enim neque, convallis ac sem a, accumsan pharetra purus. Integer quis odio libero. Mauris porttitor diam quis nisl posuere accumsan. Cras lorem tellus, condimentum at sem ac, iaculis elementum dolor. Vivamus in blandit tortor, vitae accumsan purus.

Class aptent taciti sociosqu ad litora torquent per conubia nostra, per inceptos himenaeos. Vivamus ornare urna a velit volutpat, vitae volutpat arcu semper. Nullam cursus ante sit amet purus consectetur, in aliquam nisi blandit. Etiam luctus maximus pharetra. Vivamus mattis, diam in faucibus tincidunt, lacus justo imperdiet massa, et bibendum lectus quam ut libero. Nulla et ante ac quam vehicula euismod et et felis. Mauris ac nibh congue nulla commodo dapibus. In hac habitasse platea dictumst. Maecenas consectetur justo vel convallis rutrum. Suspendisse eget tortor arcu. Pellentesque tristique fringilla elit ac placerat. Phasellus commodo mauris a accumsan aliquam. Cras tincidunt rhoncus odio. 

	\input{preambulos/3_extended_abstract}
	
	%\setstretch{1.0} % 1.0 línea de interlineado para los indices
	% Incluye después del archivo anterior el indice y lista de figuras, tablas y códigos.

	\tableofcontents     	% Índice
	\listoffigures	     	% Índice de figuras
	\listoftables	     	% Índice de tablas
	\lstlistoflistings 	    % Índice de códigos
	
	% Inicia la numeración habitual.
	\mainmatter
	
	%\setstretch{1.5} % 1.5 línea de interlineado
	
	%%%%
	% CONTENIDO. CAPÍTULOS DEL TRABAJO - Añade o elimina según tus necesidades
	%%%%
	%%%%%%%%%%%%%%%%%%%%%%%%%%%%%%%%%%%%%%%%%%%%%%%%%%%%%%%%%%%%%%%%%%%%%%%%
% Plantilla TFG/TFM
% Universidad de Murcia. Facultad de Informática
% Realizado por: José Manuel Requena Plens
% Modificado: Pablo José Rocamora Zamora
% Contacto: pablojoserocamora@gmail.com
%%%%%%%%%%%%%%%%%%%%%%%%%%%%%%%%%%%%%%%%%%%%%%%%%%%%%%%%%%%%%%%%%%%%%%%%


\chapter{Introducción}



Lorem ipsum dolor sit amet, consectetur adipiscing elit. Fusce convallis facilisis cursus. Duis convallis orci quam, sit amet lacinia massa tempus sit amet. Maecenas semper nec neque ac venenatis. Donec vehicula nibh sit amet dictum suscipit. Aenean vitae eros ut lacus molestie sagittis. Vestibulum ante ipsum primis in faucibus orci luctus et ultrices posuere cubilia Curae; Donec pellentesque fermentum ante non convallis.
\\

Ut luctus sem maximus, interdum magna ac, scelerisque sem. Morbi eu finibus dui. In ullamcorper, neque at elementum tincidunt, lectus urna accumsan tortor, gravida dictum ipsum felis vel nunc. Nam consequat faucibus urna sit amet ullamcorper. Pellentesque nec justo augue. Vivamus semper ac urna sed ultrices. Vivamus ullamcorper diam in nulla iaculis, a tincidunt risus mattis. Morbi in magna consequat, consequat dui vitae, dignissim magna. Suspendisse non turpis erat. Proin auctor enim eget erat consectetur, id semper sem porta. Proin vitae feugiat lectus. Aenean eu lacinia mauris, vitae pellentesque orci.
\\

Curabitur egestas elit non massa ultrices ultrices. Praesent sollicitudin erat ut metus lobortis, et tincidunt mi suscipit. Nunc semper, orci non mattis tristique, risus turpis feugiat enim, in lacinia arcu elit quis leo. Vivamus ultricies justo a dui pharetra, vitae condimentum erat placerat. Pellentesque vel dui fermentum, eleifend diam id, porttitor libero. Sed rutrum ac eros ut fringilla. Cras et maximus est. Ut ligula ligula, lobortis viverra magna ut, vehicula malesuada nulla. Vivamus auctor tempus nisl sit amet dapibus. Duis non aliquet augue. Morbi lacinia id urna id egestas. Nulla dictum nibh nec nulla fringilla mattis.
\\

Sed bibendum risus a nisi faucibus, eget molestie eros hendrerit. Integer at tellus vitae nulla consectetur viverra fermentum sit amet sapien. Sed volutpat, ante sagittis mattis rutrum, ligula mauris lacinia arcu, quis suscipit urna erat eu velit. Quisque ultrices vehicula purus vitae maximus. Aenean urna justo, eleifend id pellentesque a, porttitor ac ante. Sed cursus turpis lacus, sed dictum elit ultricies nec. Suspendisse a felis porttitor velit tristique tempor. Fusce vulputate, diam sit amet lobortis eleifend, nibh sapien pulvinar purus, porta maximus risus lacus sit amet lectus. Donec tincidunt neque non mi sollicitudin tincidunt. Donec porta tincidunt enim.
\\

Donec mi neque, cursus vel metus ac, rutrum faucibus turpis. Fusce sit amet arcu commodo, mollis massa at, vehicula sapien. Praesent tellus turpis, eleifend vitae suscipit vel, commodo eu justo. Nunc et risus non est pellentesque convallis. Aliquam erat volutpat. Proin iaculis libero nec mi hendrerit, sed convallis lorem aliquam. Cras posuere tempor quam rutrum feugiat. Morbi iaculis, nulla sed suscipit laoreet, mauris quam varius nunc, vel imperdiet lectus lectus eu tellus. Sed at pretium purus, eget efficitur sapien. Vivamus ac ultrices dolor, a imperdiet diam. Vestibulum ante ipsum primis in faucibus orci luctus et ultrices posuere cubilia Curae; Nullam maximus bibendum nisi, et lobortis diam eleifend nec. Proin nec tincidunt purus.
\\


Lorem ipsum dolor sit amet, consectetur adipiscing elit. Fusce convallis facilisis cursus. Duis convallis orci quam, sit amet lacinia massa tempus sit amet. Maecenas semper nec neque ac venenatis. Donec vehicula nibh sit amet dictum suscipit. Aenean vitae eros ut lacus molestie sagittis. Vestibulum ante ipsum primis in faucibus orci luctus et ultrices posuere cubilia Curae; Donec pellentesque fermentum ante non convallis.
\\

Ut luctus sem maximus, interdum magna ac, scelerisque sem. Morbi eu finibus dui. In ullamcorper, neque at elementum tincidunt, lectus urna accumsan tortor, gravida dictum ipsum felis vel nunc. Nam consequat faucibus urna sit amet ullamcorper. Pellentesque nec justo augue. Vivamus semper ac urna sed ultrices. Vivamus ullamcorper diam in nulla iaculis, a tincidunt risus mattis. Morbi in magna consequat, consequat dui vitae, dignissim magna. Suspendisse non turpis erat. Proin auctor enim eget erat consectetur, id semper sem porta. Proin vitae feugiat lectus. Aenean eu lacinia mauris, vitae pellentesque orci.
\\

Curabitur egestas elit non massa ultrices ultrices. Praesent sollicitudin erat ut metus lobortis, et tincidunt mi suscipit. Nunc semper, orci non mattis tristique, risus turpis feugiat enim, in lacinia arcu elit quis leo. Vivamus ultricies justo a dui pharetra, vitae condimentum erat placerat. Pellentesque vel dui fermentum, eleifend diam id, porttitor libero. Sed rutrum ac eros ut fringilla. Cras et maximus est. Ut ligula ligula, lobortis viverra magna ut, vehicula malesuada nulla. Vivamus auctor tempus nisl sit amet dapibus. Duis non aliquet augue. Morbi lacinia id urna id egestas. Nulla dictum nibh nec nulla fringilla mattis.
\\

Sed bibendum risus a nisi faucibus, eget molestie eros hendrerit. Integer at tellus vitae nulla consectetur viverra fermentum sit amet sapien. Sed volutpat, ante sagittis mattis rutrum, ligula mauris lacinia arcu, quis suscipit urna erat eu velit. Quisque ultrices vehicula purus vitae maximus. Aenean urna justo, eleifend id pellentesque a, porttitor ac ante. Sed cursus turpis lacus, sed dictum elit ultricies nec. Suspendisse a felis porttitor velit tristique tempor. Fusce vulputate, diam sit amet lobortis eleifend, nibh sapien pulvinar purus, porta maximus risus lacus sit amet lectus. Donec tincidunt neque non mi sollicitudin tincidunt. Donec porta tincidunt enim.
\\

Donec mi neque, cursus vel metus ac, rutrum faucibus turpis. Fusce sit amet arcu commodo, mollis massa at, vehicula sapien. Praesent tellus turpis, eleifend vitae suscipit vel, commodo eu justo. Nunc et risus non est pellentesque convallis. Aliquam erat volutpat. Proin iaculis libero nec mi hendrerit, sed convallis lorem aliquam. Cras posuere tempor quam rutrum feugiat. Morbi iaculis, nulla sed suscipit laoreet, mauris quam varius nunc, vel imperdiet lectus lectus eu tellus. Sed at pretium purus, eget efficitur sapien. Vivamus ac ultrices dolor, a imperdiet diam. Vestibulum ante ipsum primis in faucibus orci luctus et ultrices posuere cubilia Curae; Nullam maximus bibendum nisi, et lobortis diam eleifend nec. Proin nec tincidunt purus. 
\\

Curabitur egestas elit non massa ultrices ultrices. Praesent sollicitudin erat ut metus lobortis, et tincidunt mi suscipit. Nunc semper, orci non mattis tristique, risus turpis feugiat enim, in lacinia arcu elit quis leo. Vivamus ultricies justo a dui pharetra, vitae condimentum erat placerat. Pellentesque vel dui fermentum, eleifend diam id, porttitor libero. Sed rutrum ac eros ut fringilla. Cras et maximus est. Ut ligula ligula, lobortis viverra magna ut, vehicula malesuada nulla. Vivamus auctor tempus nisl sit amet dapibus. Duis non aliquet augue. Morbi lacinia id urna id egestas. Nulla dictum nibh nec nulla fringilla mattis.
\\

Sed bibendum risus a nisi faucibus, eget molestie eros hendrerit. Integer at tellus vitae nulla consectetur viverra fermentum sit amet sapien. Sed volutpat, ante sagittis mattis rutrum, ligula mauris lacinia arcu, quis suscipit urna erat eu velit. Quisque ultrices vehicula purus vitae maximus. Aenean urna justo, eleifend id pellentesque a, porttitor ac ante. Sed cursus turpis lacus, sed dictum elit ultricies nec. Suspendisse a felis porttitor velit tristique tempor. Fusce vulputate, diam sit amet lobortis eleifend, nibh sapien pulvinar purus, porta maximus risus lacus sit amet lectus. Donec tincidunt neque non mi sollicitudin tincidunt. Donec porta tincidunt enim.
\\

Donec mi neque, cursus vel metus ac, rutrum faucibus turpis. Fusce sit amet arcu commodo, mollis massa at, vehicula sapien. Praesent tellus turpis, eleifend vitae suscipit vel, commodo eu justo. Nunc et risus non est pellentesque convallis. Aliquam erat volutpat. Proin iaculis libero nec mi hendrerit, sed convallis lorem aliquam. Cras posuere tempor quam rutrum feugiat. Morbi iaculis, nulla sed suscipit laoreet, mauris quam varius nunc, vel imperdiet lectus lectus eu tellus. Sed at pretium purus, eget efficitur sapien. Vivamus ac ultrices dolor, a imperdiet diam. Vestibulum ante ipsum primis in faucibus orci luctus et ultrices posuere cubilia Curae; Nullam maximus bibendum nisi, et lobortis diam eleifend nec. Proin nec tincidunt purus. 


Sed bibendum risus a nisi faucibus, eget molestie eros hendrerit. Integer at tellus vitae nulla consectetur viverra fermentum sit amet sapien. Sed volutpat, ante sagittis mattis rutrum, ligula mauris lacinia arcu, quis suscipit urna erat eu velit. Quisque ultrices vehicula purus vitae maximus. Aenean urna justo, eleifend id pellentesque a, porttitor ac ante. Sed cursus turpis lacus, sed dictum elit ultricies nec. Suspendisse a felis porttitor velit tristique tempor. Fusce vulputate, diam sit amet lobortis eleifend, nibh sapien pulvinar purus, porta maximus risus lacus sit amet lectus. Donec tincidunt neque non mi sollicitudin tincidunt. Donec porta tincidunt enim.
\\

Donec mi neque, cursus vel metus ac, rutrum faucibus turpis. Fusce sit amet arcu commodo, mollis massa at, vehicula sapien. Praesent tellus turpis, eleifend vitae suscipit vel, commodo eu justo. Nunc et risus non est pellentesque convallis. Aliquam erat volutpat. Proin iaculis libero nec mi hendrerit, sed convallis lorem aliquam. Cras posuere tempor quam rutrum feugiat. Morbi iaculis, nulla sed suscipit laoreet, mauris quam varius nunc, vel imperdiet lectus lectus eu tellus. Sed at pretium purus, eget efficitur sapien. Vivamus ac ultrices dolor, a imperdiet diam. Vestibulum ante ipsum primis in faucibus orci luctus et ultrices posuere cubilia Curae; Nullam maximus bibendum nisi, et lobortis diam eleifend nec. Proin nec tincidunt purus. 
\\

Curabitur egestas elit non massa ultrices ultrices. Praesent sollicitudin erat ut metus lobortis, et tincidunt mi suscipit. Nunc semper, orci non mattis tristique, risus turpis feugiat enim, in lacinia arcu elit quis leo. Vivamus ultricies justo a dui pharetra, vitae condimentum erat placerat. Pellentesque vel dui fermentum, eleifend diam id, porttitor libero. Sed rutrum ac eros ut fringilla. Cras et maximus est. Ut ligula ligula, lobortis viverra magna ut, vehicula malesuada nulla. Vivamus auctor tempus nisl sit amet dapibus. Duis non aliquet augue. Morbi lacinia id urna id egestas. Nulla dictum nibh nec nulla fringilla mattis.

Sed bibendum risus a nisi faucibus, eget molestie eros hendrerit. Integer at tellus vitae nulla consectetur viverra fermentum sit amet sapien. Sed volutpat, ante sagittis mattis rutrum, ligula mauris lacinia arcu, quis suscipit urna erat eu velit. Quisque ultrices vehicula purus vitae maximus. Aenean urna justo, eleifend id pellentesque a, porttitor ac ante. Sed cursus turpis lacus, sed dictum elit ultricies nec. Suspendisse a felis porttitor velit tristique tempor. Fusce vulputate, diam sit amet lobortis eleifend, nibh sapien pulvinar purus, porta maximus risus lacus sit amet lectus. Donec tincidunt neque non mi sollicitudin tincidunt. Donec porta tincidunt enim.
\\

Donec mi neque, cursus vel metus ac, rutrum faucibus turpis. Fusce sit amet arcu commodo, mollis massa at, vehicula sapien. Praesent tellus turpis, eleifend vitae suscipit vel, commodo eu justo. Nunc et risus non est pellentesque convallis. Aliquam erat volutpat. Proin iaculis libero nec mi hendrerit, sed convallis lorem aliquam. Cras posuere tempor quam rutrum feugiat. Morbi iaculis, nulla sed suscipit laoreet, mauris quam varius nunc, vel imperdiet lectus lectus eu tellus. Sed at pretium purus, eget efficitur sapien. Vivamus ac ultrices dolor, a imperdiet diam. Vestibulum ante ipsum primis in faucibus orci luctus et ultrices posuere cubilia Curae; Nullam maximus bibendum nisi, et lobortis diam eleifend nec. Proin nec tincidunt purus. 
\\


Sed bibendum risus a nisi faucibus, eget molestie eros hendrerit. Integer at tellus vitae nulla consectetur viverra fermentum sit amet sapien. Sed volutpat, ante sagittis mattis rutrum, ligula mauris lacinia arcu, quis suscipit urna erat eu velit. Quisque ultrices vehicula purus vitae maximus. Aenean urna justo, eleifend id pellentesque a, porttitor ac ante. Sed cursus turpis lacus, sed dictum elit ultricies nec. Suspendisse a felis porttitor velit tristique tempor. Fusce vulputate, diam sit amet lobortis eleifend, nibh sapien pulvinar purus, porta maximus risus lacus sit amet lectus. Donec tincidunt neque non mi sollicitudin tincidunt. Donec porta tincidunt enim.
\\

Donec mi neque, cursus vel metus ac, rutrum faucibus turpis. Fusce sit amet arcu commodo, mollis massa at, vehicula sapien. Praesent tellus turpis, eleifend vitae suscipit vel, commodo eu justo. Nunc et risus non est pellentesque convallis. Aliquam erat volutpat. Proin iaculis libero nec mi hendrerit, sed convallis lorem aliquam. Cras posuere tempor quam rutrum feugiat. Morbi iaculis, nulla sed suscipit laoreet, mauris quam varius nunc, vel imperdiet lectus lectus eu tellus. Sed at pretium purus, eget efficitur sapien. Vivamus ac ultrices dolor, a imperdiet diam. Vestibulum ante ipsum primis in faucibus orci luctus et ultrices posuere cubilia Curae; Nullam maximus bibendum nisi, et lobortis diam eleifend nec. Proin nec tincidunt purus. 
\\

Curabitur egestas elit non massa ultrices ultrices. Praesent sollicitudin erat ut metus lobortis, et tincidunt mi suscipit. Nunc semper, orci non mattis tristique, risus turpis feugiat enim, in lacinia arcu elit quis leo. Vivamus ultricies justo a dui pharetra, vitae condimentum erat placerat. Pellentesque vel dui fermentum, eleifend diam id, porttitor libero. Sed rutrum ac eros ut fringilla. Cras et maximus est. Ut ligula ligula, lobortis viverra magna ut, vehicula malesuada nulla. Vivamus auctor tempus nisl sit amet dapibus. Duis non aliquet augue. Morbi lacinia id urna id egestas. Nulla dictum nibh nec nulla fringilla mattis.
\\

Sed bibendum risus a nisi faucibus, eget molestie eros hendrerit. Integer at tellus vitae nulla consectetur viverra fermentum sit amet sapien. Sed volutpat, ante sagittis mattis rutrum, ligula mauris lacinia arcu, quis suscipit urna erat eu velit. Quisque ultrices vehicula purus vitae maximus. Aenean urna justo, eleifend id pellentesque a, porttitor ac ante. Sed cursus turpis lacus, sed dictum elit ultricies nec. Suspendisse a felis porttitor velit tristique tempor. Fusce vulputate, diam sit amet lobortis eleifend, nibh sapien pulvinar purus, porta maximus risus lacus sit amet lectus. Donec tincidunt neque non mi sollicitudin tincidunt. Donec porta tincidunt enim.
\\

Donec mi neque, cursus vel metus ac, rutrum faucibus turpis. Fusce sit amet arcu commodo, mollis massa at, vehicula sapien. Praesent tellus turpis, eleifend vitae suscipit vel, commodo eu justo. Nunc et risus non est pellentesque convallis. Aliquam erat volutpat. Proin iaculis libero nec mi hendrerit, sed convallis lorem aliquam. Cras posuere tempor quam rutrum feugiat. Morbi iaculis, nulla sed suscipit laoreet, mauris quam varius nunc, vel imperdiet lectus lectus eu tellus. Sed at pretium purus, eget efficitur sapien. Vivamus ac ultrices dolor, a imperdiet diam. Vestibulum ante ipsum primis in faucibus orci luctus et ultrices posuere cubilia Curae; Nullam maximus bibendum nisi, et lobortis diam eleifend nec. Proin nec tincidunt purus. 
\\


Sed bibendum risus a nisi faucibus, eget molestie eros hendrerit. Integer at tellus vitae nulla consectetur viverra fermentum sit amet sapien. Sed volutpat, ante sagittis mattis rutrum, ligula mauris lacinia arcu, quis suscipit urna erat eu velit. Quisque ultrices vehicula purus vitae maximus. Aenean urna justo, eleifend id pellentesque a, porttitor ac ante. Sed cursus turpis lacus, sed dictum elit ultricies nec. Suspendisse a felis porttitor velit tristique tempor. Fusce vulputate, diam sit amet lobortis eleifend, nibh sapien pulvinar purus, porta maximus risus lacus sit amet lectus. Donec tincidunt neque non mi sollicitudin tincidunt. Donec porta tincidunt enim.
\\

Donec mi neque, cursus vel metus ac, rutrum faucibus turpis. Fusce sit amet arcu commodo, mollis massa at, vehicula sapien. Praesent tellus turpis, eleifend vitae suscipit vel, commodo eu justo. Nunc et risus non est pellentesque convallis. Aliquam erat volutpat. Proin iaculis libero nec mi hendrerit, sed convallis lorem aliquam. Cras posuere tempor quam rutrum feugiat. Morbi iaculis, nulla sed suscipit laoreet, mauris quam varius nunc, vel imperdiet lectus lectus eu tellus. Sed at pretium purus, eget efficitur sapien. Vivamus ac ultrices dolor, a imperdiet diam. Vestibulum ante ipsum primis in faucibus orci luctus et ultrices posuere cubilia Curae; Nullam maximus bibendum nisi, et lobortis diam eleifend nec. Proin nec tincidunt purus. 
\\

Curabitur egestas elit non massa ultrices ultrices. Praesent sollicitudin erat ut metus lobortis, et tincidunt mi suscipit. Nunc semper, orci non mattis tristique, risus turpis feugiat enim, in lacinia arcu elit quis leo. Vivamus ultricies justo a dui pharetra, vitae condimentum erat placerat. Pellentesque vel dui fermentum, eleifend diam id, porttitor libero. Sed rutrum ac eros ut fringilla. Cras et maximus est. Ut ligula ligula, lobortis viverra magna ut, vehicula malesuada nulla. Vivamus auctor tempus nisl sit amet dapibus. Duis non aliquet augue. Morbi lacinia id urna id egestas. Nulla dictum nibh nec nulla fringilla mattis.
\\

Sed bibendum risus a nisi faucibus, eget molestie eros hendrerit. Integer at tellus vitae nulla consectetur viverra fermentum sit amet sapien. Sed volutpat, ante sagittis mattis rutrum, ligula mauris lacinia arcu, quis suscipit urna erat eu velit. Quisque ultrices vehicula purus vitae maximus. Aenean urna justo, eleifend id pellentesque a, porttitor ac ante. Sed cursus turpis lacus, sed dictum elit ultricies nec. Suspendisse a felis porttitor velit tristique tempor. Fusce vulputate, diam sit amet lobortis eleifend, nibh sapien pulvinar purus, porta maximus risus lacus sit amet lectus. Donec tincidunt neque non mi sollicitudin tincidunt. Donec porta tincidunt enim.
\\

Donec mi neque, cursus vel metus ac, rutrum faucibus turpis. Fusce sit amet arcu commodo, mollis massa at, vehicula sapien. Praesent tellus turpis, eleifend vitae suscipit vel, commodo eu justo. Nunc et risus non est pellentesque convallis. Aliquam erat volutpat. Proin iaculis libero nec mi hendrerit, sed convallis lorem aliquam. Cras posuere tempor quam rutrum feugiat. Morbi iaculis, nulla sed suscipit laoreet, mauris quam varius nunc, vel imperdiet lectus lectus eu tellus. Sed at pretium purus, eget efficitur sapien. Vivamus ac ultrices dolor, a imperdiet diam. Vestibulum ante ipsum primis in faucibus orci luctus et ultrices posuere cubilia Curae; Nullam maximus bibendum nisi, et lobortis diam eleifend nec. Proin nec tincidunt purus. 	
	\input{capitulos/2_estado_arte}
	

\chapter{Análisis de objetivos y metodología} 

	

\chapter{Diseño y resolución del trabajo realizado} 

	\input{capitulos/5_conclusiones}
	\input{capitulos/_introduccion}		% Plantilla: Se muestran contenidos
	\input{capitulos/_marcoteorico}		% Plantilla: Se muestran listas
	\input{capitulos/_objetivos}		% Plantilla: Se muestran tablas
	\input{capitulos/_metodologia}		% Plantilla: Se muestran figuras
	\input{capitulos/_desarrollo}		% Plantilla: Se muestran listados
	\input{capitulos/_resultados}		% Plantilla: Se muestran gráficas
	\input{capitulos/_conclusiones}		% Plantilla: Se muestran matemáticas
	
	%%%%
	% CONTENIDO. BIBLIOGRAFÍA.
	%%%%
	\nocite{*} %incluye TODOS los documentos de la base de datos bibliográfica sean o no citados en el texto
	\bibliographystyle{unsrtnat}
	\bibliography{bibliografia/bibliografia} % Archivo que contiene la bibliografía
	
	
	%%%%
	% CONTENIDO. LISTA DE ACRÓNIMOS. Comenta las líneas si no lo deseas incluir.
	%%%%
	% Incluye el listado de acrónimos utilizados en el trabajo. 
	\printglossary[style=modsuper,type=\acronymtype,title={Lista de Acrónimos y Abreviaturas}]
	% Añade el resto de acrónimos si así se desea. Si no elimina el comando siguiente
	\glsaddallunused
	
	%%%%
	% CONTENIDO. Anexos - Añade o elimina según tus necesidades
	%%%%
	\appendix % Inicio de los apéndices
	\input{anexos/_anexo_I}
	\input{anexos/_anexo_2}
	\input{anexos/_anexo_3}
	
\end{document}