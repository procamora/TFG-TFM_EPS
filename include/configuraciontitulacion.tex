%%%%%%%%%%%%%%%%%%%%%%%%%%%%%%%%%%%%%%%%%%%%%%%%%%%%%%%%%%%%%%%%%%%%%%%%
% Plantilla TFG/TFM
% Escuela Politécnica Superior de la Universidad de Alicante
% Realizado por: Jose Manuel Requena Plens
% Contacto: info@jmrplens.com / Telegram:@jmrplens
%%%%%%%%%%%%%%%%%%%%%%%%%%%%%%%%%%%%%%%%%%%%%%%%%%%%%%%%%%%%%%%%%%%%%%%%

%%%%%%%%%%%%%%%%%%%%%%%% 
% COLORES DE GRADOS.
% Si el color de la titulación ha cambiado, modifícalo en las lineas siguientes.
%%%%%%%%%%%%%%%%%%%%%%%%
% Grados
\definecolor{informatica}{RGB}{121,11,21}	% Informatica

% Logotipos comunes de todas las titulaciones

\newcommand{\logoFacultadPortada}{include/logos-titulaciones/Simbolo-negativo-sin-fondo}
\newcommand{\logoGradoPortada}{include/logos-titulaciones/Simbolo-negativo-sin-fondo} 
\newcommand{\logoUniversidadPortada}{include/logos-universidad/escudo_umu}
\newcommand{\logoFacultadPortadaBaja}{include/logos-titulaciones/FIUM-postivo-sin-fondo}

% Colores generales
\definecolor{negro}{RGB}{0,0,0}
\definecolor{blanco}{RGB}{255,255,255}
%%%%%%%%%%%%%%%%%%%%%%%% 
% CONDICIONALES. SEGUN LA ID ELEGIDA EN EL .TEX PRINCIPAL
% Según el ID seleccionado en TFG_EPS_UA.tex se configurará el nombre de la titulación, logotipos y color.
% Si tu titulación no esta correctamente definida cambia las imágenes que se definen para tu titulación en las lineas de abajo
% Si deseas añadir mas titulaciones ve al final de este archivo
%%%%%%%%%%%%%%%%%%%%%%%%

% Logos
\newcommand{\logoGrado}{include/logos-titulaciones/FI-positivo}
\newcommand{\logoDepartamento}{include/logos-titulaciones/Logotipo_DIIC}
\newcommand{\logoUniversidad}{include/logos-universidad/um}
% Texto
\newcommand{\miGrado}{Grado en Ingeniería Informática}
\newcommand{\tipotrabajo}{Trabajo Fin de Grado}
% Color
\newcommand{\colorgrado}{informatica}
\newcommand{\colortexto}{blanco}
	
	
%%%%%%%%%%%%%%%%%%%%%%%%%%%%%%%%%%%%%%%%%%%%%%%%%%%%%%%%%%%%%%%%%%%%%%%%	
% ¿COMO AÑADIR MÁS TITULACIONES?
% Para añadir más titulaciones, se debe continuar el el formato de ID -> Titulacion.
% Justo encima de la linea donde hay muchos '\fi' se debe escribir el condicional y el contenido de este tal que:
%
%	\else \if\IDtitulo X % Titulacion con ID=X		
% 		% Logos
%		\newcommand{\logoFacultadPortada}{include/logos-universidad/LogoEPSBlanco}
%		\newcommand{\logoGradoPortada}{include/logos-titulaciones/logotitulacion}
%		\newcommand{\logoGrado}{include/logos-titulaciones/logotitulacion}
%		% Texto
%		\newcommand{\miGrado}{Grado en XXXXXXXX}
%		\newcommand{\tipotrabajo}{Trabajo Fin de XXXX}
%		% Color
%		\newcommand{\colorgrado}{XXXX}
%		\newcommand{\colortexto}{XXX}
%	
% Por último añadir a la linea que tiene muchos '\fi', otro '\fi'. Listo, ya podrás usar la nueva ID con la configuración añadida.
%%%%%%%%%%%%%%%%%%%%%%%%%%%%%%%%%%%%%%%%%%%%%%%%%%%%%%%%%%%%%%%%%%%%%%%%	




