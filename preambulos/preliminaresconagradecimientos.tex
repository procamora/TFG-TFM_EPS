%%%%%%%%%%%%%%%%%%%%%%%%%%%%%%%%%%%%%%%%%%%%%%%%%%%%%%%%%%%%%%%%%%%%%%%%
% Plantilla TFG/TFM
% Universidad de Murcia. Facultad de Informática
% Realizado por: José Manuel Requena Plens
% Modificado: Pablo José Rocamora Zamora
% Contacto: pablojoserocamora@gmail.com
%%%%%%%%%%%%%%%%%%%%%%%%%%%%%%%%%%%%%%%%%%%%%%%%%%%%%%%%%%%%%%%%%%%%%%%%


\chapter*{Preámbulo}
\thispagestyle{empty}
Poner aquí un texto breve que debe incluir entre otras:
\begin{quote}
	las razones que han llevado a la realización del estudio, el tema, la finalidad y el alcance y también los agradecimientos por las ayudas, por ejemplo apoyo económico (becas y subvenciones) y las consultas y discusiones con los tutores y colegas de trabajo. \citep{UNE50136:97}
\end{quote}


%\cleardoublepage %salta a nueva página impar

\chapter*{Agradecimientos\footnote{Por si alguien tiene curiosidad, este ``simpático'' agradecimiento está tomado de la ``Tesis de Lydia Chalmers'' basada en el universo del programa de televisión Buffy, la Cazadora de Vampiros.http://www.buffy-cazavampiros.com/Spiketesis/tesis.inicio.htm}
}

\thispagestyle{empty}
\vspace{1cm}

Este trabajo no habría sido posible sin el apoyo y el estímulo de mi colega y amigo, Doctor Rudolf Fliesning,  bajo cuya supervisión escogí este tema y comencé la tesis. Sr. Quentin Travers, mi consejero en las etapas finales del trabajo, también ha sido generosamente servicial, y me ha ayudado de numerosos modos, incluyendo el resumen del contenido de los documentos que no estaban disponibles para mi examen, y en particular por permitirme leer, en cuanto estuvieron  disponibles, las copias de los  recientes extractos de los diarios de campaña del Vigilante Rupert Giles y la actual Cazadora la señorita Buffy Summers, que se encontraron con William the Bloody en 1998, y por facilitarme el pleno acceso  a los diarios de anteriores Vigilantes relevantes a la carrera de William the Bloody.

También me gustaría agradecerle al Consejo la concesión de Wyndham-Pryce como Compañero, el cual me ha apoyado durante mis dos años de investigación, y la concesión de dos subvenciones de viajes, una para estudiar documentos en los Archivos de Vigilantes sellados en Munich, y otra para la investigación en campaña en Praga. Me gustaría agradecer a Sr. Travers, otra vez, por facilitarme  la acreditación  de seguridad para el trabajo en los Archivos de Munich, y al Doctor Fliesning por su apoyo colegial y ayuda en ambos viajes de investigación.

No puedo terminar sin agradecer a mi familia, en cuyo estímulo constante y amor he confiado a lo largo de mis años en la Academia. Estoy agradecida también a los ejemplos de mis  difuntos hermano, Desmond Chalmers, Vigilante en Entrenamiento, y padre, Albert Chalmers, Vigilante. Su coraje resuelto y convicción siempre me inspirarán, y espero seguir, a mi propio y pequeño modo, la noble misión por la que dieron sus vidas. 

Es a ellos a quien dedico este trabajo.

%\cleardoublepage %salta a nueva página impar

% Aquí va la dedicatoria si la hubiese. Si no, comentar la(s) linea(s) siguientes
\chapter*{}
\setlength{\leftmargin}{0.5\textwidth}
\setlength{\parsep}{0cm}
\addtolength{\topsep}{0.5cm}

\begin{flushright}
	\small\em{
		A mi esposa Marganit, y a mis hijos Ella Rose y Daniel Adams,\\
		sin los cuales habría podido acabar este libro dos años antes \footnote{Dedicatoria de Joseph J. Roman en "An Introduction to Algebraic Topology"}
	}
\end{flushright}


%\cleardoublepage %salta a nueva página impar

% Aquí va la cita célebre si la hubiese. Si no, comentar la(s) linea(s) siguientes
\chapter*{}

\setlength{\leftmargin}{0.5\textwidth}
\setlength{\parsep}{0cm}
\addtolength{\topsep}{0.5cm}
\begin{flushright}
	\small\em{
		Si consigo ver más lejos\\
		es porque he conseguido auparme\\ 
		a hombros de gigantes
	}
\end{flushright}

\begin{flushright}
	\small{
		Isaac Newton.
	}
\end{flushright}

%\cleardoublepage %salta a nueva página impar

